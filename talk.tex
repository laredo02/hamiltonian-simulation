
\documentclass{article}
\usepackage[a4paper, margin=2cm]{geometry}
\usepackage{amsmath, amssymb, amsthm}

\begin{document}

\section{The Hamiltonian}
We consider a Hamiltonian acting on $n$ qubits:
\begin{equation}
  H = \alpha (s_1 \otimes s_2 \otimes \dots \otimes s_n)
\end{equation}
where $s_i$ are Pauli matrices ($X, Y, Z$), and $\alpha$ is a scalar. Our goal is to implement the time evolution operator efficiently:
\begin{equation}
  e^{-itH} = e^{-it\alpha (s_1 \otimes s_2 \otimes \dots \otimes s_n)}
\end{equation}

\section{The Challenge}
\begin{itemize}
\item $H$ is a $2^n \times 2^n$ matrix, making direct exponentiation impractical.
\item Quantum circuits use polynomially many gates ($O(\text{poly}(n))$), so we need an efficient approach.
\end{itemize}

\section{Unitary Transformation Trick}
We use the key identity:
\begin{equation}
  U e^{-i\alpha H} U^\dagger = e^{-i\alpha UHU^\dagger}.
\end{equation}

This follows from the definition of the matrix exponential.

\begin{proof}
  Expanding the power series:
  \begin{equation}
    e^{-i\alpha H} = \sum_{k=0}^{\infty} \frac{(-i\alpha H)^k}{k!},
  \end{equation}
  and applying conjugation by $U$:
  \begin{equation}
    U e^{-i\alpha H} U^\dagger = U \left( \sum_{k=0}^{\infty} \frac{(-i\alpha H)^k}{k!} \right) U^\dagger.
  \end{equation}
  Using $UHU^\dagger = H'$, we get:
  \begin{equation}
    U e^{-i\alpha H} U^\dagger = \sum_{k=0}^{\infty} \frac{(-i\alpha UHU^\dagger)^k}{k!} = e^{-i\alpha H'}.
  \end{equation}
  
\end{proof}

\section{Diagonalizing $H$}

\begin{equation}
X = \begin{pmatrix} 0 & 1 \\ 1 & 0 \end{pmatrix}, \quad
Y = \begin{pmatrix} 0 & -i \\ i & 0 \end{pmatrix}, \quad
Z = \begin{pmatrix} 1 & 0 \\ 0 & -1 \end{pmatrix}
\end{equation}

\begin{itemize}

    \item \textbf{Pauli-X diagonalization}:
    \begin{equation}
    HXH = Z
    \end{equation}
    The Hadamard gate $H$ transforms $X$ to diagonal $Z$ form.

    \item \textbf{Pauli-Y diagonalization}:
    \begin{equation}
    U_Y Y U_Y^\dagger = Z \quad \text{where} \quad U_Y = \frac{1}{\sqrt{2}}\begin{pmatrix} 1 & 1 \\ i & -i \end{pmatrix}
    \end{equation}
    The unitary $U_Y$ transforms $Y$ to diagonal $Z$ form.
  \end{itemize}


  \section{Putting it all together}

Applying these transformations to all qubits, we rewrite the evolution operator as:
\begin{equation}
  e^{-itH} = (U_1 \otimes \dots \otimes U_n) e^{-i\alpha t (z_1 \otimes \dots \otimes z_n)} (U_1^\dagger \otimes \dots \otimes U_n^\dagger),
\end{equation}
where $z_i$ are now either $I$ or $Z$, making the Hamiltonian diagonal.

\section{Why This Helps}
By diagonalizing $H$, we reduce the problem to implementing:
\begin{equation}
  e^{-i\alpha t Z \otimes Z \otimes \dots \otimes Z},
\end{equation}
which is simple because:
\begin{itemize}
\item Pauli $Z$ matrices are diagonal, making exponentiation trivial, example of diagonalization.
\item The circuit needs only controlled phase gates, requiring $O(n)$ quantum gates.
\end{itemize}

\section{Conclusion}
Using unitary transformations, we efficiently simulate the time evolution of a tensor-product Pauli Hamiltonian in polynomial time, making it feasible for quantum computation.

\end{document}
